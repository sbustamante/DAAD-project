\documentclass[a4,useAMS,usenatbib,usegraphicx,12pt]{article}
%External Packages and personalized macros
\include{latex/macros}

 
\title{{\textbf{Research Proposal for DAAD PhD scholarship}}\\ The Gaseous Cosmic Web \\ \color{black}\rule{15cm}{0.5mm}}
\author{Sebastian Bustamante Jaramillo}
\date{}
 
\definecolor{lightgray}{gray}{0.8}
\newcolumntype{L}{>{\raggedleft}p{0.14\textwidth}}
\newcolumntype{R}{p{0.8\textwidth}}
\newcommand\VRule{\color{lightgray}\vrule width 0.5pt}
  
\begin{document}
\maketitle
\tableofcontents
 
\newpage 

%============================================================================== 
\section{General Information}
\small
\subsection*{Information of the Applicant}
\begin{tabular}{L!{\VRule}R}
\bf Name		& Sebastian Bustamante Jaramillo\\
\bf Degree		& B.Sc. in Physics, Universidad de Antioquia (2012)\\
\bf Position	& Adjunct Professor, Universidad de Antioquia\\
\bf Birthday	& { 20$^{th}$ June, 1990}\\
\bf Nationality & Colombian\\
\bf ID			& C.C. 1128400433\\
\bf Address 1	& Avenida 21 \# 57 AA 65, Bello - Colombia (personal)\\
\bf Address 2	& Calle 67 \# 53 - 108, Off. 5-330, Medellin, Colombia (work)\\
\bf Phone		& +057 (4) 4820138\\
\bf Mobile		& +057 3108992409\\
\bf E-mail 1	& macsebas33 \textit{at} gmail.com (personal)\\
\bf E-mail 2	& sebastian.bustamante \textit{at} udea.edu.co (academic)\\
\end{tabular}

\vspace{10pt}

More detailed information of the applicant can be found here \url{http://goo.gl/BPZGzK}

\vspace{15pt}  

\subsection*{Information of the Project}
\begin{tabular}{L!{\VRule}R}
\bf Title		& \bf The Gaseous Cosmic Web\\
\bf Field		& Cosmology, Astrophysics, Physical Sciences \\
\bf Advisor 1	& Volker Springel, PhD. Heidelberg Institute for Theoretical Studies (HITS) \& University of Heidelberg, Germany \\
\bf Advisor 2	& Jaime Forero-Romero, PhD. Universidad de los Andes, Colombia \\
\bf University	& University of Heidelberg, IMPRS PhD program \\
\bf Time Frame	& 3 years \\
\end{tabular}
\normalsize
%==============================================================================


%==============================================================================
\section{Abstract}
%==============================================================================


%==============================================================================
\section{Introduction}
Since the filamentary nature of the large-scale matter distribution of the 
observable cosmos (the so-called cosmic web) was evidenced by the first compiled 
galaxy surveys \citep{Chincarini75, Gregory78, Einasto80M, Einasto80N, 
Kirshner81, Kirshner87}, it has been identified as one of the most striking 
features of the Megaparsec Universe and an increasing interest in studying its 
dynamical properties and environmental influences on a plethora of different 
astrophysical phenomena has become evident. At present, a tremendous amount of 
observational data supports the cosmic web scenario at the point that it has 
become an essential part of the current standard paradigm in cosmology. Last 
generation galaxy redshift surveys, such as the \textit{two-degree-Field Galaxy 
Redshift Survey} (2dFGRS) and the \textit{Sloan Digital Sky Survey} (SDSS), do
evince the intricate and complex structure of the cosmic web at a level of 
detail never seen before. In addition, other valuable observational resources 
like X-ray emissions of hot intracluster gas embedded into large clusters of 
galaxies, Ly-$\alpha$ forest absorption lines in the spectra of shock-heated 
neutral hydrogen gas residing in filaments and clusters, and weak gravitational 
lensing and imprints in the CMB field produced by foreground structures have 
also validated this picture undoubtedly.



On the theoretical side, early descriptions of the evolution of the large-scale 
Universe, based on gravitational instabilities in primordial stages and leaded 
by the seminal work of \citet{Zeldovich70}, are highly consistent with the 
cosmic web picture, where planar pancake-like regions of matter enclose enormous
sub-dense voids and are bordered, in turn, by thin filaments and high-density 
clumpy knots \citep{Bond96}. Since then, our understanding of the structure and 
dynamics of the cosmic web has been dramatically improved as new and more 
powerful theoretical and computational tools and more refined observational data 
become available. In particular, N-body simulations, fuelled by last generation 
computing systems and ever more efficient numerical algorithms, are acquiring 
an increasingly important role in fathoming the complexity of the large-scale 
Universe.



Due to the poorly interacting nature of the dark matter component of the 
cosmic inventory, observations have been devoted to establish the 
underlying structure of the cosmic web entirely based on detecting baryonic 
matter (with the exception of non-direct inferences based on weak gravitational
lensing). On the other hand, the highly complex \textit{gastrophysical} 
processes involved in baryonic dynamics, i.e. shock heating, photoionization, 
supernova feedback, stellar wind, radiative cooling, star formation and others, 
make extremely difficult to obtain a completely consistent and reliable scenario 
from numerical simulations as many of these processes are not fully understood 
yet. Accordingly, most of the related numerical research has been made based on 
dark matter-only N-body simulations, where the gas dynamics has been neglected.
Although this duality between observations and simulations can be thought as a 
complementary situation, actually it also makes quite hard to splice both, 
observational data and numerical predictions.



Although a visual naked-eye identification of single structures (i.e. voids, 
walls, filaments or knots) is always possible in galaxy surveys and simulations,
translating this visual impression into a formal mathematical formulation is 
not such easy and much effort has been made in this direction. 


%==============================================================================


%==============================================================================
\section{Objectives}
%==============================================================================


%==============================================================================
\section{Methods}
%==============================================================================


%==============================================================================
\section{Expected Results}
%==============================================================================


%==============================================================================
\section{Bibliography}
%==============================================================================
\bibliographystyle{latex/mn2e}
\bibliography{references}


%==============================================================================
\section{Timetable}
%==============================================================================


\end{document}
\documentclass[a4,useAMS,usenatbib,usegraphicx,12pt]{article}
%External Packages and personalized macros
%=========================================================================
%		EXTERNAL PACKAGES
%=========================================================================
\usepackage[round]{natbib}
\usepackage[margin=3cm]{geometry}
\usepackage{hyperref}
\usepackage{times}
\usepackage{amsmath} 
\usepackage{amssymb}
\usepackage{graphicx}
\usepackage{array, xcolor, lipsum, bibentry}
\usepackage[nottoc, notlof, notlot]{tocbibind}

\definecolor{lightgray}{gray}{0.8}
\newcolumntype{L}{>{\raggedleft}p{0.14\textwidth}}
\newcolumntype{R}{p{0.8\textwidth}}
\newcommand\VRule{\color{lightgray}\vrule width 0.5pt}

\usepackage{booktabs}% http://ctan.org/pkg/booktabs
\newcommand{\tabitem}{~~\llap{\textbullet}~~}

%=========================================================================
%		INTERNAL MACROS
%=========================================================================
% To highlight comments 
\definecolor{red}{rgb}{1,0.0,0.0}
\newcommand{\red}{\color{red}}
\definecolor{darkgreen}{rgb}{0.0,0.5,0.0}
\newcommand{\SRK}[1]{\textcolor{darkgreen}{\bf SRK: \textit{#1}}}
\newcommand{\SRKED}[1]{\textcolor{darkgreen}{\bf #1}}

\newcommand{\LCDM}{$\Lambda$CDM~}
\newcommand{\beq}{\begin{eqnarray}}  
\newcommand{\eeq}{\end{eqnarray}}  
\newcommand{\zz}{$z\sim 3$} 
\newcommand{\apj}{ApJ}  
\newcommand{\apjs}{ApJS}  
\newcommand{\apjl}{ApJL}  
\newcommand{\aj}{AJ}  
\newcommand{\mnras}{MNRAS}  
\newcommand{\mnrassub}{MNRAS accepted}  
\newcommand{\aap}{A\&A}  
\newcommand{\aaps}{A\&AS}  
\newcommand{\araa}{ARA\&A}  
\newcommand{\nat}{Nature}  
\newcommand{\physrep}{PhR}
\newcommand{\pasp}{PASP}    
\newcommand{\pasj}{PASJ}    
\newcommand{\avg}[1]{\langle{#1}\rangle}  
\newcommand{\ly}{{\ifmmode{{\rm Ly}\alpha}\else{Ly$\alpha$}\fi}}
\newcommand{\hMpc}{{\ifmmode{h^{-1}{\rm Mpc}}\else{$h^{-1}$Mpc }\fi}}  
\newcommand{\hGpc}{{\ifmmode{h^{-1}{\rm Gpc}}\else{$h^{-1}$Gpc }\fi}}  
\newcommand{\hmpc}{{\ifmmode{h^{-1}{\rm Mpc}}\else{$h^{-1}$Mpc }\fi}}  
\newcommand{\hkpc}{{\ifmmode{h^{-1}{\rm kpc}}\else{$h^{-1}$kpc }\fi}}  
\newcommand{\hMsun}{{\ifmmode{h^{-1}{\rm {M_{\odot}}}}\else{$h^{-1}{\rm{M_{\odot}}}$}\fi}}  
\newcommand{\hmsun}{{\ifmmode{h^{-1}{\rm {M_{\odot}}}}\else{$h^{-1}{\rm{M_{\odot}}}$}\fi}}  
\newcommand{\Msun}{{\ifmmode{{\rm {M_{\odot}}}}\else{${\rm{M_{\odot}}}$}\fi}}  
\newcommand{\msun}{{\ifmmode{{\rm {M_{\odot}}}}\else{${\rm{M_{\odot}}}$}\fi}}  
\newcommand{\lya}{{Lyman$\alpha$~}}
\newcommand{\clara}{{\texttt{CLARA}}~}
\newcommand{\rand}{{\ifmmode{{\mathcal{R}}}\else{${\mathcal{R}}$ }\fi}}  


%MY COMMANDS #############################################################
\newcommand{\sub}[1]{\mbox{\scriptsize{#1}}}
\newcommand{\dtot}[2]{ \frac{ d #1 }{d #2} }
\newcommand{\dpar}[2]{ \frac{ \partial #1 }{\partial #2} }
\newcommand{\pr}[1]{ \left( #1 \right) }
\newcommand{\corc}[1]{ \left[ #1 \right] }
\newcommand{\lla}[1]{ \left\{ #1 \right\} }
\newcommand{\bds}[1]{\boldsymbol{ #1 }}
\newcommand{\oiint}{\displaystyle\bigcirc\!\!\!\!\!\!\!\!\int\!\!\!\!\!\int}
\newcommand{\mathsize}[2]{\mbox{\fontsize{#1}{#1}\selectfont $#2$}}
\newcommand{\eq}[2]{\begin{equation} \label{eq:#1} #2 \end{equation}}
\newcommand{\lth}{$\lambda_{th}$ }
%#########################################################################

\setlength\parindent{0pt}
 
\title{{\textbf{Research Proposal for DAAD PhD scholarship}}\\ The Gaseous Cosmic Web with AREPO\\ \color{black}\rule{15cm}{0.5mm}}
\author{Sebastian Bustamante Jaramillo}
\date{}
  
\begin{document}
\maketitle
\begin{center}
\includegraphics[trim = 0mm 16cm 0mm 20cm, clip, keepaspectratio=true,
width=0.7\textheight]{Presentation2.png}
\tiny{A projection of the cosmic web in the ILLUSTRIS simulation, that was made 
with AREPO (http://www.illustris-project.org/)}
\end{center}
\tableofcontents
 
\newpage 

%============================================================================== 
\section{General Information}
\small
\subsection*{Information of the Applicant}
\begin{tabular}{L!{\VRule}R}
\bf Name		& Sebastian Bustamante Jaramillo\\
\bf Degree		& B.Sc. in Physics, Universidad de Antioquia (2012)\\
\bf Position	& Adjunct Professor, Universidad de Antioquia\\
\bf Birthday	& { 20$^{th}$ June, 1990}\\
\bf Nationality & Colombian\\
\bf ID			& C.C. 1128400433\\
\bf Address 1	& Avenida 21 \# 57 AA 65, Bello - Colombia (personal)\\
\bf Address 2	& Calle 67 \# 53 - 108, Off. 5-330, Medellin, Colombia (work)\\
\bf Phone		& +057 (4) 4820138\\
\bf Mobile		& +057 3108992409\\
\bf E-mail 1	& macsebas33 \textit{at} gmail.com (personal)\\
\bf E-mail 2	& sebastian.bustamante \textit{at} udea.edu.co (academic)\\
\end{tabular}

\vspace{10pt}

More detailed information of the applicant can be found here \url{http://goo.gl/BPZGzK}

\vspace{15pt}  

\subsection*{Information of the Project}
\begin{tabular}{L!{\VRule}R}
\bf Title		& \bf The Gaseous Cosmic Web with AREPO\\
\bf Field		& Cosmology, Astrophysics, Physical Sciences \\
\bf Advisor 1	& Volker Springel, PhD. Heidelberg Institute for Theoretical Studies (HITS) \& University of Heidelberg, Germany \\
\bf Advisor 2	& Jaime Forero-Romero, PhD. Universidad de los Andes, Colombia \\
\bf University	& University of Heidelberg, IMPRS PhD program \\
\bf Time Frame	& 3 years \\
\end{tabular}
\normalsize
%==============================================================================


%==============================================================================
\section{Abstract}
%==============================================================================


%==============================================================================
\section{Introduction}


Since the filamentary nature of the large-scale matter distribution of the 
observable cosmos (the so-called cosmic web) was evidenced by the first compiled 
galaxy surveys \citep{Chincarini75, Gregory78, Einasto80M, Einasto80N, 
Kirshner81, Kirshner87}, it has been identified as one of the most striking 
features of the Megaparsec Universe and an increasing interest in studying its 
dynamical properties and environmental influences on a plethora of different 
astrophysical phenomena has become evident. At present, a tremendous amount of 
observational data supports the cosmic web scenario at the point that it has 
become an essential part of the current standard paradigm in cosmology. Last 
generation galaxy redshift surveys, such as the \textit{two-degree-Field Galaxy 
Redshift Survey} (2dFGRS) and the \textit{Sloan Digital Sky Survey} (SDSS), do
evince the intricate and complex structure of the cosmic web at a level of 
detail never seen before. In addition, other valuable observational resources 
like X-ray emissions of hot intracluster gas embedded into large clusters of 
galaxies, Ly-$\alpha$ forest absorption lines in the spectra of shock-heated 
neutral hydrogen gas residing in filaments and clusters, and weak gravitational 
lensing and imprints in the CMB field produced by foreground structures, have 
also validated this picture undoubtedly.

\

On the theoretical side, early descriptions of the evolution of the large-scale 
Universe, based on gravitational instabilities in primordial stages and leaded 
by the seminal work of \citet{Zeldovich70}, are highly consistent with the 
cosmic web picture, where planar pancake-like regions of matter enclose enormous
sub-dense voids and are bordered, in turn, by thin filaments and high-density 
clumpy knots \citep{Bond96}. Since then, our understanding of the structure and 
dynamics of the cosmic web has been dramatically improved as new and more 
powerful theoretical and computational tools and more refined observational data 
become available. In particular, N-body simulations, fuelled by last generation 
computing systems and ever more efficient numerical algorithms, are acquiring 
an increasingly important role in fathoming the complexity of the large-scale 
Universe.

\

Due to the poorly interacting (and unknown) nature of the dark matter component 
of the  cosmic inventory, observations have been devoted to establish the 
underlying structure of the cosmic web entirely based on detecting baryonic 
matter (with the exception of non-direct inferences based on gravitational 
lensing). On the other hand, the highly complex \textit{gastrophysical} 
processes involved in baryonic dynamics, i.e. shock heating, photoionization, 
supernova feedback, stellar wind, radiative cooling, star formation and others, 
make extremely difficult to obtain a completely consistent and reliable scenario 
from numerical simulations as many of these processes are not fully understood 
yet. Accordingly, most of the related numerical research has been made based on 
dark matter-only N-body simulations, where the gas dynamics has been neglected.
Although this duality between observations and simulations can be thought as a 
complementary situation, actually it also makes quite hard to splice both, 
observational data and numerical predictions.

\

In spite of most of the above-mentioned \textit{gastrophysical} processes 
occurring in baryonic dynamics do represent a challenge for any endeavour for 
simulating the large-scale Universe, the merely hydrodynamic nature of the gas
has been challenging enough even for the most simplified models. Traditionally,
two different hydro-solvers has been used for astrophysical and 
cosmological applications, i.e. \textit{Adaptive Mesh Refinement} (AMR) and 
\textit{Smoothed Particle Hydrodynamics} (SPH).


%==============================================================================

\newpage
%==============================================================================
\section{Objectives}
%==============================================================================


\begin{itemize}

\item[\checkmark] Understanding, at the light of the last generation AREPO 
simulations, how is the dynamics of the baryonic matter component throughout
the pipeline set up by the potential wells of the dark matter cosmic web.

\item[\checkmark] Quantifying gaseous environmental effects on galaxy formation
and dynamics.

\item[\checkmark] Comparing our results with current (predicting new) observables 
and imprints of the gaseous cosmic web.


\end{itemize}



%==============================================================================
\section{Methodology}
%==============================================================================


The proposed project is subject to a PhD study and will cover the following 
aspects:

\

\begin{itemize}

\item[\checkmark] \textit{First, a set of simulations must be established for 
all succeeding steps. This may involve making new hydrodynamical simulations or 
adopting existing ones based on the AREPO code. This also includes a complete 
analysis of the simulations, i.e. characterization of physical fields (density, 
temperature, entropy), construction and analysis of catalogues of haloes, and
others.}

\end{itemize}

As this project will be almost entirely based on numerical results, establishing
a set of precise hydrodynamical simulations as a solid base for all our succeeding
studies is, indeed, one of the key steps that must be fulfilled. The unprecedented 
accuracy and convergence achieved by the AREPO code, regarding other traditionally 
used schemes, will guarantee the needed precision.


In Heidelberg, the required computer facilities and the access to the private 
code AREPO (of which Prof. Volker Springel is the main author) is granted.

\

\begin{itemize}

\item[\checkmark] \textit{Second, an preliminary exploration of the dark matter 
cosmic web should be done. For this purpose, it is necessary an adaptation of 
some commonly used web-finding schemes (e.g. the V-web and the T-web) to the 
new Voronoi-based paradigm established by the AREPO code.}

\end{itemize}


A first exploration of the structures of the simulations should be done. For 
this purpose, many different schemes can be found in the literature, 
but taking into account the reported success of tensor-based 
web-finding schemes (e.g. the V-web and the T-web), in which prof. Jaime 
Forero-Romero has a wide research trajectory, it is quite interesting to 
quantify the dark matter cosmic web of last generation simulations at the light 
of them. Nevertheless, it is also necessary to adapt these schemes to the 
Voronoi-based paradigm leaded by AREPO as they was originally intended for 
traditionally used Cloud-in-Cell methods for computing the density field.
It is also worth to mention previous research experiences in this topic of the
applicant.

\

\begin{itemize}

\item[\checkmark] \textit{Third, once established the underlying dark matter 
cosmic web, it is necessary to quantify the through gas dynamics. To this aim, 
inward and outward gas fluxes through potential wells (set by non-linear 
structures such as clusters and filaments) and accretion rates of the gas 
component residing within dark matter halos at different redshifts should be 
computed. At this part, it is enabled to evaluate environmental influences on
different astrophysical phenomena. }

\end{itemize}


In order to exploit the new accuracy provide by the AREPO code, a correct 
quantification of the gas dynamics should be done. The current cosmological 
paradigm predicts a complex pipeline set by the potential well of non-linear 
structures, generally corresponding to clusters and filaments, through which 
the gas is transported toward over-dense regions like dark matter haloes. This
process yields to different environmental phenomena of great current interest.
We list here the most relevant for this project: influence of filament-induced 
fluxes on star forming galaxies at high redshifts, acquisition of the spin of 
galaxies through exchanging of angular momentum with the gaseous cosmic web, 
and kinematical and dynamical effects of the host environment on Local 
Group-like systems.

\

\begin{itemize}

\item[\checkmark] \textit{Finally, a detailed comparison of our potential 
predictions (or restrictions) with currently available observational data
should be done.}

\end{itemize}



%==============================================================================
\section{Current State}
%==============================================================================


At present, the applicant has already the fundamental basis in Astrophysics and
Cosmology required for this investigation. This can be confirmed by his research 
trajectory, including a paper (as co-author) published in \textit{ApJL} in which 
was studied the kinematics of the Local Group in a cosmological context, and some
participations in academic congresses. Furthermore, a Bachelor's thesis
\footnote{Further information and an electronic version of this 
thesis can be found here \url{https://github.com/sbustamante/Thesis}.}, where was 
numerically studied the preferred place of Local Group-like systems regarding the 
host environment in the cosmic web, also proves the ability of the applicant for
handling simulations and massive data, a skill that is indeed necessary for 
carrying out this project.

\

A deeper training in paralleling computing is still needed, however the applicant
is already working on it.


%==============================================================================
\section{Timetable}
%==============================================================================

\begin{table}[h]
\begin{flushleft}
\begin{center}
  \begin{tabular}{l  l} \hline\hline
	\centering\textbf{Year} & \textbf{Goals} \\ \hline
	%First year
	First  
	& \tabitem First goal \\
	& \tabitem Second goal\\
	
	\\
	%Second year	
	Second
	& \tabitem First goal \\
	& \tabitem Second goal\\

	\\	
	%Third year	
	Third
	& \tabitem First goal \\
	& \tabitem Second goal\\ 
	
	\hline\hline
  \end{tabular}  
\end{center}
\end{flushleft}
\end{table}


%==============================================================================
\bibliographystyle{latex/mn2e}
\renewcommand{\bibname}{8\ \ \ \ Bibliography}
\bibliography{references.bib}
%==============================================================================



\end{document}
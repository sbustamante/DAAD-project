\documentclass[a4,useAMS,usenatbib,usegraphicx,12pt]{article}
%External Packages and personalized macros
%=========================================================================
%		EXTERNAL PACKAGES
%=========================================================================
\usepackage[round]{natbib}
\usepackage[margin=3cm]{geometry}
\usepackage{hyperref}
\usepackage{times}
\usepackage{amsmath} 
\usepackage{amssymb}
\usepackage{graphicx}
\usepackage{array, xcolor, lipsum, bibentry}
\usepackage[nottoc, notlof, notlot]{tocbibind}

\definecolor{lightgray}{gray}{0.8}
\newcolumntype{L}{>{\raggedleft}p{0.14\textwidth}}
\newcolumntype{R}{p{0.8\textwidth}}
\newcommand\VRule{\color{lightgray}\vrule width 0.5pt}

\usepackage{booktabs}% http://ctan.org/pkg/booktabs
\newcommand{\tabitem}{~~\llap{\textbullet}~~}

%=========================================================================
%		INTERNAL MACROS
%=========================================================================
% To highlight comments 
\definecolor{red}{rgb}{1,0.0,0.0}
\newcommand{\red}{\color{red}}
\definecolor{darkgreen}{rgb}{0.0,0.5,0.0}
\newcommand{\SRK}[1]{\textcolor{darkgreen}{\bf SRK: \textit{#1}}}
\newcommand{\SRKED}[1]{\textcolor{darkgreen}{\bf #1}}

\newcommand{\LCDM}{$\Lambda$CDM~}
\newcommand{\beq}{\begin{eqnarray}}  
\newcommand{\eeq}{\end{eqnarray}}  
\newcommand{\zz}{$z\sim 3$} 
\newcommand{\apj}{ApJ}  
\newcommand{\apjs}{ApJS}  
\newcommand{\apjl}{ApJL}  
\newcommand{\aj}{AJ}  
\newcommand{\mnras}{MNRAS}  
\newcommand{\mnrassub}{MNRAS accepted}  
\newcommand{\aap}{A\&A}  
\newcommand{\aaps}{A\&AS}  
\newcommand{\araa}{ARA\&A}  
\newcommand{\nat}{Nature}  
\newcommand{\physrep}{PhR}
\newcommand{\pasp}{PASP}    
\newcommand{\pasj}{PASJ}    
\newcommand{\avg}[1]{\langle{#1}\rangle}  
\newcommand{\ly}{{\ifmmode{{\rm Ly}\alpha}\else{Ly$\alpha$}\fi}}
\newcommand{\hMpc}{{\ifmmode{h^{-1}{\rm Mpc}}\else{$h^{-1}$Mpc }\fi}}  
\newcommand{\hGpc}{{\ifmmode{h^{-1}{\rm Gpc}}\else{$h^{-1}$Gpc }\fi}}  
\newcommand{\hmpc}{{\ifmmode{h^{-1}{\rm Mpc}}\else{$h^{-1}$Mpc }\fi}}  
\newcommand{\hkpc}{{\ifmmode{h^{-1}{\rm kpc}}\else{$h^{-1}$kpc }\fi}}  
\newcommand{\hMsun}{{\ifmmode{h^{-1}{\rm {M_{\odot}}}}\else{$h^{-1}{\rm{M_{\odot}}}$}\fi}}  
\newcommand{\hmsun}{{\ifmmode{h^{-1}{\rm {M_{\odot}}}}\else{$h^{-1}{\rm{M_{\odot}}}$}\fi}}  
\newcommand{\Msun}{{\ifmmode{{\rm {M_{\odot}}}}\else{${\rm{M_{\odot}}}$}\fi}}  
\newcommand{\msun}{{\ifmmode{{\rm {M_{\odot}}}}\else{${\rm{M_{\odot}}}$}\fi}}  
\newcommand{\lya}{{Lyman$\alpha$~}}
\newcommand{\clara}{{\texttt{CLARA}}~}
\newcommand{\rand}{{\ifmmode{{\mathcal{R}}}\else{${\mathcal{R}}$ }\fi}}  


%MY COMMANDS #############################################################
\newcommand{\sub}[1]{\mbox{\scriptsize{#1}}}
\newcommand{\dtot}[2]{ \frac{ d #1 }{d #2} }
\newcommand{\dpar}[2]{ \frac{ \partial #1 }{\partial #2} }
\newcommand{\pr}[1]{ \left( #1 \right) }
\newcommand{\corc}[1]{ \left[ #1 \right] }
\newcommand{\lla}[1]{ \left\{ #1 \right\} }
\newcommand{\bds}[1]{\boldsymbol{ #1 }}
\newcommand{\oiint}{\displaystyle\bigcirc\!\!\!\!\!\!\!\!\int\!\!\!\!\!\int}
\newcommand{\mathsize}[2]{\mbox{\fontsize{#1}{#1}\selectfont $#2$}}
\newcommand{\eq}[2]{\begin{equation} \label{eq:#1} #2 \end{equation}}
\newcommand{\lth}{$\lambda_{th}$ }
%#########################################################################

\setlength\parindent{0pt}
 
\title{{\textbf{Extension Report}}\\ 
				DAAD PhD Scholarship\\ 
				\color{black}\rule{15cm}{0.5mm}}
\author{Sebastian Bustamante Jaramillo}
\date{}
  
\begin{document}
\maketitle

\tableofcontents
 
\newpage 

%============================================================================== 
\section{Report}
%============================================================================== 

Since my arrival in Germany to the present day it has been almost one year. This
time can be split-up into two periods. The first four months, from 06.15 to 09.15,
correspond to the German course in Berlin. Finally, the second period, from 10.15 
to 04.16, correspond to my arrival in Heidelberg for starting my PhD.

\

Regarding the first period, there is not much academic progress to report on as 
I was focused on learning German. The only two exceptions perhaps were a 
first-author paper I pu\-blished in 08.15 \citep{Bustamante15}
\footnote{\url{http://mnras.oxfordjournals.org/content/453/1/497}} and an invited
talk I gave at AIP (Leibniz-Institut f\"ur Astrophysik Potsdam) about the paper. 
Nevertheless, all the results reported in this work were already developed during 
my studies in Colombia and the activities during my stay in Berlin were limited 
to corrections of referee reports.

\

For the second period there are several things to report on. These can be split-up 
in 3 categories: general, corresponding to activities such as taking lectures 
and seminars, getting the admission to the university, etc. Major project, which 
is related to the main project I am developing with my supervisor, and finally, 
a minor project developed with a postdoc of the group.


%------------------------------------------------------------------------------
\subsection{General}
%------------------------------------------------------------------------------ 

After my arrival in Heidelberg, the first step was to enrol to the University as
a PhD student, however the assessment emitted by the HGFSP office (Heidelberg 
Graduate School of Fundamental Physics) was to take preparatory studies before 
being admitted as PhD student (this was because I do not hold a master's degree), 
which means I had to take two core lectures of the master. Nevertheless, this 
assessment was out of date as it corresponds to my old application in 2013. Since 
then, I had taken 1 semester of master studies in Colombia and published a 
first-author paper. Unfortunately, I had not brought documents to prove this, so
I had to wait for someone to bring them to me from Colombia. In the meantime,
it was decided by my Thesis Committee that I should take one of the core lectures 
during the first semester, so I decided to take Theoretical Astrophysics.

At the end of the semester, I finally got the documents, so I immediately 
submitted them to the HGFSP office and one week later I was officially admitted 
as a PhD student in the University of Heidelberg. It is worth mentioning that I 
also finished the lecture, which included homework and a final examination.

\

Although the astronomy PhD program is officially hosted by the HGFSP, the 
associated activities are organized by the MPIA (Max Planck Institute for 
Astronomy) within the framework of the IMPRS program (International Max Planck
Research School), which is a structured program, and for that reason, there are 
several activities that the student members are required to participate in. 
Among these activities are the IMPRS literature seminar and the first-year 
student retreat. The seminar corresponds to a weekly session where every student 
must give a short talk about an assigned research paper. In my case, I attended 
this seminar during the first semester and also gave a talk about diffusive 
shock acceleration\footnote{\url{https://www.youtube.com/watch?v=L7OVGV6XJ0k}}.
About the first-year retreat, it consists of a social integration for new PhD
students and a visit to some astronomy institute in some city out of Germany.
The last retreat took place on 13.04.16 (four days) in Budapest, Hungary, where
we visited the Konkoly Observatory.


%------------------------------------------------------------------------------
\subsection{Major Project}
%------------------------------------------------------------------------------

The initial project I submitted in 2014 for the application to the scholarship 
was about the study of the gaseous cosmic web in the context of the state-of-art 
hydrodynamical simulations using AREPO code \citep{Springel2010} (e.g. Illustris 
simulation\footnote{\url{http://www.illustris-project.org/}}). However, after
my arrival in Heidelberg and during my first meeting with my supervisor, it was 
decided to change the focus of the main project to study dynamical friction of
supermassive black holes in the centres of galaxies in order to make predictions 
of gravitational-wave emissions in a cosmological context. The main reason for 
this decision was that black hole physics was going to become a very trendy topic 
in the light of the forthcoming gravitational wave experiments (e.g. LIGO 
\footnote{Laser Interferometer Gravitational-Wave Observatory, 
\url{https://www.ligo.caltech.edu/}}), and robust predictions are going to be 
needed from theoretical and numerical models. Nowadays, with the confirmed discovery 
of gravitational waves by the LIGO team, the change of course proved to be a very 
wise judgement.

With the new project already defined, I proceeded to compile and read references
about the topic, and, along with my supervisor, to write a new research proposal
to be submitted to the DAAD (this document is attached in the appendices).

\

In the first two months, I was focused on comprehending the process of dynamical 
friction and how it operates in the context of galaxy dynamics. This was an
important step because this process drives the orbital decay of supermassive 
black holes, specially after a galaxy merger, where two or more black holes 
coalesce after emitting a burst of gravitational waves. For the purpose of 
understanding this process, I developed some scripts in python and a small 
N-body code\footnote{\url{https://github.com/sbustamante/N-body}} to study how 
numerical resolution effects influence the trajectory of a massive body embedded 
in some background medium composed of slightly lighter particles, which is 
indeed a fair approximation to the real problem faced when simulating 
supermassive black holes in galaxies.

\

Once this initial stage was successfully done, I moved on to use the 
hydrodynamic code AREPO to simulate isolated galaxies and merger events (see
figure \ref{fig:simulations}), which, although yet simplistic, were closer 
to the original problem that we want to solve. The objective with this 
exercise was to get familiar with the code and understand the problems with 
the current treatment of supermassive black holes \citep{Springel11}, where 
their position are glued to the potential minimum of the galaxy. 
This approach has proven to be numerically stable, yet unrealistic.


%.........................................................................
%FIGURE 1: Galaxies
\begin{figure}[!htbp]
\centering

  \includegraphics[trim = 0mm 0mm 0mm 0mm, clip, keepaspectratio=true,
  width=0.60\textheight]{./figures/Galaxies.png}
  
  \caption{\small Left panel: isolated galaxy with a supermassive black 
  hole in the centre (red dot). Right panel: a merger of two galaxies 
  after a first passage. Two black holes are shown in the centre of each
  galaxy.}

  \label{fig:simulations}

\end{figure}
%.........................................................................

\

The last three months were dedicated to devise an improved treatment for the 
dynamical friction experienced by supermassive black holes in numerical 
simulations. In this regard, the main problem emerges when one tries to simulate
a continuous stars and gas background using large sampling particles, with a size 
comparable to the black hole. This produces unrealistic orbits, where stochastic 
numerical heating prevails rather than a smooth decay as expected from dynamical
friction driven trajectories. The described procedure is unfortunately necessary 
due to the limited computing power achieved by modern computers. That is the 
reason why sub-grid semi-analytic models are always required. 

\

Several approaches have been proposed throughout the literature, namely glueing 
the position of the black hole to the potential minimum, as mentioned above. 
Using an improved version of the Chandrasekhar formula, where only the closer 
neighbour particles are used to compute the distribution function of the 
background \citep{Tremmel2015}. We also introduced and implemented (in AREPO)
our own approach, where an effective drag force is estimated assuming an 
overdamped decay which follows the velocity of the potential minimum. Although 
this method is only partially physically motivated, it is indeed an effective 
way to yield decaying orbits, even for low resolutions.

\

During this time, I also devised a set of numerical experiments to test our 
approach. It consists of simulating a black hole particle embedded in an 
isolated dark matter halo that follows a Hernquist profile 
\citep{Hernquist1990}. This has two important advantages: firstly, this profile 
is stationary, meaning that there is not transient or numerical relaxation, 
and secondly, it has analytical profiles for the density, the potential and 
the velocity distribution function. Both features facilitate enormously 
comparison between numerical results and analytical predictions.


%.........................................................................
%FIGURE 2: Dynamical friction decay
\begin{figure}[!htbp]
\centering

  \includegraphics[trim = 0mm 0mm 0mm 0mm, clip, keepaspectratio=true,
  width=0.7\textheight]{./figures/Results.png}
  
  \caption{\small Left panel: numerical simulations of the decay of a 
  supermassive black hole in a circular orbit, without using our approach
  (black line), and using it (grey line). Right panel: evolution of the
  radial component for the orbit of the black hole, using the numerical
  orbits (black and grey lines) and analytical integrations for a pure
  Chandrasekhar friction force (red line) and for a overdamped drag force
  (blue line).}

  \label{fig:dynamical_friction}

\end{figure}
%.........................................................................


It is shown in figure \ref{fig:dynamical_friction} the results of our numerical
integration and the corresponding analytical predictions. It is worth noting 
that we were using a rather high resolution ($1\times 10^6$ particles), so it 
is expected that dynamical friction is accounted for at some extent. Indeed, a 
decay is observed for the case without an effective drag force and the orbit 
is close to the one predicted analytically using the Chandrasekhar formula with 
a correction for the maximum impact parameter \citep{Just2011}. In the case 
when the drag force is activated, the numerical result and the analytical 
prediction are different. However, the numerical orbit exhibits a faster decay 
and is less noisy once it gets close to the centre. Further study in this direction 
is necessary to fine tune our drag friction force to reproduce realistic decay 
times.

\

Finally, a last problem we are currently working on is related to tracking 
the trajectory of the potential minimum of a galaxy, which is an essential step 
for our approach as the drag force is precisely given from that frame of 
reference. The method currently in use consists of following the most bound 
particle. Nevertheless, due to numerical noise, this has shown to be numerically 
unstable, producing levels of noise comparable with the resolution effects that 
we want precisely to deal with. The alternative that is being now developed 
consists of fitting the potential field around the black hole using a least 
squares multidimensional minimization with the neighbour particles, which might
offer a more stable solution for the potential minimum trajectory. Once this
approach, along with the effective drag force, are working, our short-term plans 
are to run a cosmological simulation and write a first paper about the results.


\newpage
%------------------------------------------------------------------------------
\subsection{Minor Project}
%------------------------------------------------------------------------------

Additionally, a minor project has been going on during the previous semester, 
which is being developed with a postdoc of the group at HITS (Martin Sparre).
The goal of this project is to study the effect of minor mergers on the
star formation rate of galaxies at different redshifts. This is important in
the context of galaxy formation because high star formation rates in galaxies 
at high redshifts remain unexplained. So far, there are several hypothesis 
supporting this picture, namely, accretion of cold star forming gas through 
infalling filamentary flows which are able to cross trough the shock-heated 
regions around galaxies and penetrate towards the centre. A second hypothesis 
consists of major merger events, where tidal forces compress the gas and 
trigger star formation. Finally, we propose to study a third scenario, that is 
not exclusive with the other two, but offers an appealing alternative as minor 
merger events are very common and not as potentially destructive as major 
mergers.

%.........................................................................
%FIGURE 3: Star formation rate
\begin{figure}[!htbp]
\centering

  \includegraphics[trim = 0mm 0mm 0mm 0mm, clip, keepaspectratio=true,
  width=0.7\textheight]{./figures/SFR.pdf}
  
  \caption{\small Star formation history for a galaxy in the Auriga project.
  The peaks are associated to star bursts induced by major or minor mergers.}

  \label{fig:SFR}

\end{figure}
%.........................................................................

For the purpose of this project, we use a set of zoom cosmological simulations 
(the Auriga project), where several galaxies have been simulated with very high
resolutions from sub-boxes of the Illustris simulation (see more details in
\citet{Grand16})). So far, several analysis scripts have been developed using 
the library MergerZoomAnalysis\footnote{
\url{https://bitbucket.org/martinsparre/mergerzoomanalysis}} of Martin Sparre. 
For example, in figure \ref{fig:SFR} it is shown the star formation history of 
one of the simulated galaxies, where the peaks correspond to star bursts induced 
by minor or major mergers. My current objective is then to develop a method to 
track minor merger events and correlate them with star bursts in the star 
formation history.


%==============================================================================
\bibliographystyle{latex/mn2e}
\renewcommand{\bibname}{8\ \ \ \ Bibliography}
\small
\bibliography{references2.bib}
%==============================================================================



\end{document}

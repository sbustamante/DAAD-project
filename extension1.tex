\documentclass[a4,useAMS,usenatbib,usegraphicx,12pt]{article}
%External Packages and personalized macros
%=========================================================================
%		EXTERNAL PACKAGES
%=========================================================================
\usepackage[round]{natbib}
\usepackage[margin=3cm]{geometry}
\usepackage{hyperref}
\usepackage{times}
\usepackage{amsmath} 
\usepackage{amssymb}
\usepackage{graphicx}
\usepackage{array, xcolor, lipsum, bibentry}
\usepackage[nottoc, notlof, notlot]{tocbibind}

\definecolor{lightgray}{gray}{0.8}
\newcolumntype{L}{>{\raggedleft}p{0.14\textwidth}}
\newcolumntype{R}{p{0.8\textwidth}}
\newcommand\VRule{\color{lightgray}\vrule width 0.5pt}

\usepackage{booktabs}% http://ctan.org/pkg/booktabs
\newcommand{\tabitem}{~~\llap{\textbullet}~~}

%=========================================================================
%		INTERNAL MACROS
%=========================================================================
% To highlight comments 
\definecolor{red}{rgb}{1,0.0,0.0}
\newcommand{\red}{\color{red}}
\definecolor{darkgreen}{rgb}{0.0,0.5,0.0}
\newcommand{\SRK}[1]{\textcolor{darkgreen}{\bf SRK: \textit{#1}}}
\newcommand{\SRKED}[1]{\textcolor{darkgreen}{\bf #1}}

\newcommand{\LCDM}{$\Lambda$CDM~}
\newcommand{\beq}{\begin{eqnarray}}  
\newcommand{\eeq}{\end{eqnarray}}  
\newcommand{\zz}{$z\sim 3$} 
\newcommand{\apj}{ApJ}  
\newcommand{\apjs}{ApJS}  
\newcommand{\apjl}{ApJL}  
\newcommand{\aj}{AJ}  
\newcommand{\mnras}{MNRAS}  
\newcommand{\mnrassub}{MNRAS accepted}  
\newcommand{\aap}{A\&A}  
\newcommand{\aaps}{A\&AS}  
\newcommand{\araa}{ARA\&A}  
\newcommand{\nat}{Nature}  
\newcommand{\physrep}{PhR}
\newcommand{\pasp}{PASP}    
\newcommand{\pasj}{PASJ}    
\newcommand{\avg}[1]{\langle{#1}\rangle}  
\newcommand{\ly}{{\ifmmode{{\rm Ly}\alpha}\else{Ly$\alpha$}\fi}}
\newcommand{\hMpc}{{\ifmmode{h^{-1}{\rm Mpc}}\else{$h^{-1}$Mpc }\fi}}  
\newcommand{\hGpc}{{\ifmmode{h^{-1}{\rm Gpc}}\else{$h^{-1}$Gpc }\fi}}  
\newcommand{\hmpc}{{\ifmmode{h^{-1}{\rm Mpc}}\else{$h^{-1}$Mpc }\fi}}  
\newcommand{\hkpc}{{\ifmmode{h^{-1}{\rm kpc}}\else{$h^{-1}$kpc }\fi}}  
\newcommand{\hMsun}{{\ifmmode{h^{-1}{\rm {M_{\odot}}}}\else{$h^{-1}{\rm{M_{\odot}}}$}\fi}}  
\newcommand{\hmsun}{{\ifmmode{h^{-1}{\rm {M_{\odot}}}}\else{$h^{-1}{\rm{M_{\odot}}}$}\fi}}  
\newcommand{\Msun}{{\ifmmode{{\rm {M_{\odot}}}}\else{${\rm{M_{\odot}}}$}\fi}}  
\newcommand{\msun}{{\ifmmode{{\rm {M_{\odot}}}}\else{${\rm{M_{\odot}}}$}\fi}}  
\newcommand{\lya}{{Lyman$\alpha$~}}
\newcommand{\clara}{{\texttt{CLARA}}~}
\newcommand{\rand}{{\ifmmode{{\mathcal{R}}}\else{${\mathcal{R}}$ }\fi}}  


%MY COMMANDS #############################################################
\newcommand{\sub}[1]{\mbox{\scriptsize{#1}}}
\newcommand{\dtot}[2]{ \frac{ d #1 }{d #2} }
\newcommand{\dpar}[2]{ \frac{ \partial #1 }{\partial #2} }
\newcommand{\pr}[1]{ \left( #1 \right) }
\newcommand{\corc}[1]{ \left[ #1 \right] }
\newcommand{\lla}[1]{ \left\{ #1 \right\} }
\newcommand{\bds}[1]{\boldsymbol{ #1 }}
\newcommand{\oiint}{\displaystyle\bigcirc\!\!\!\!\!\!\!\!\int\!\!\!\!\!\int}
\newcommand{\mathsize}[2]{\mbox{\fontsize{#1}{#1}\selectfont $#2$}}
\newcommand{\eq}[2]{\begin{equation} \label{eq:#1} #2 \end{equation}}
\newcommand{\lth}{$\lambda_{th}$ }
%#########################################################################

\setlength\parindent{0pt}
 
\title{{\textbf{Extension Report}}\\ 
				DAAD PhD Scholarship\\ 
				\color{black}\rule{15cm}{0.5mm}}
\author{Sebastian Bustamante Jaramillo}
\date{}
  
\begin{document}
\maketitle

\tableofcontents
 
\newpage 

%============================================================================== 
\section{Report}
%============================================================================== 

Since my arrival in Germany to the present day it has been almost one year. This
time can be split-up into two periods. The first four months, from 06.15 to 09.15,
correspond to the German course in Berlin. Finally, the second period, from 10.15 
to 04.16, correspond to my arrival in Heidelberg for starting my PhD.

\

Regarding the first period, there is not much academic progress to report on as 
I was focused on learning German. The only two exceptions perhaps were a 
first-author paper I pu\-blished in 08.15 \citep{Bustamante15}
\footnote{\url{http://mnras.oxfordjournals.org/content/453/1/497}} and an invited
talk I gave at AIP (Leibniz-Institut f\"ur Astrophysik Potsdam) about the paper. 
Nevertheless, all the results reported in this work were already developed during 
my studies in Colombia and the activities during my stay in Berlin were limited 
to corrections of referee reports.

\

For the second period there are several things to report on. These can be split-up 
in 3 categories: general, corresponding to activities such as taking lectures 
and seminars, getting the admission to the university, etc. Major project, which 
is related to the main project I am developing with my supervisor, and finally, 
a minor project developed with a postdoc of the group.


%------------------------------------------------------------------------------
\subsection{General}
%------------------------------------------------------------------------------ 

After my arrival in Heidelberg, the first step was to enrol to the University as
a PhD student, however the assessment emitted by the HGFSP office (Heidelberg 
Graduate School of Fundamental Physics) was to take preparatory studies before 
being admitted as PhD student (this was because I do not hold a master's degree), 
which means I had to take two core lectures of the master. Nevertheless, this 
assessment was out of date as it corresponds to my old application in 2013. Since 
then, I had taken 1 semester of master studies in Colombia and published a 
first-author paper. Unfortunately, I had not brought documents to prove this, so
I had to wait for someone to bring them to me from Colombia. In the meantime,
it was decided by my Thesis Committee that I should take one of the core lectures 
during the first semester, so I decided to take Theoretical Astrophysics.

At the end of the semester, I finally got the documents, so I immediately 
submitted to the HGFSP office and one week later I was officially admitted as a
PhD student in the University of Heidelberg. It is worth mentioning that I also
finished the lecture, which included homework and a final examination.

\

Although the astronomy PhD program is officially hosted by the HGFSP, the 
associated activities are organized by the MPIA (Max Planck Institute for 
Astronomy) within the framework of the IMPRS program (International Max Planck
Research School), which is a structured program, and for that reason, there are 
several activities that the student members are required to participate in. 
Among these activities are the IMPRS literature seminar and the first-year 
student retreat. The seminar corresponds to a weekly session where every student 
must give a short talk about an assigned research paper. In my case, I attended 
this seminar during the first semester and also gave a talk about diffusive 
shock acceleration\footnote{\url{https://www.youtube.com/watch?v=L7OVGV6XJ0k}}.
About the first-year retreat, it consists of a social integration for new PhD
students and a visit to some astronomy institute in some city out of Germany.
The last retreat took place on 13.04.16 (four days) in Budapest, Hungary, where
we visited the Konkoly Observatory.


%------------------------------------------------------------------------------
\subsection{Major Project}
%------------------------------------------------------------------------------

The initial project I submitted in 2014 when applying for the scholarship was
about the study of the gaseous cosmic web in the context of the state-of-art 
hydrodynamical simulations using AREPO code (e.g. Illustris simulation\footnote{
\url{http://www.illustris-project.org/}}).



%------------------------------------------------------------------------------
\subsection{Minor Project}
%------------------------------------------------------------------------------

Additionally, a minor project has been going on during the previous semester, 
which is being developed with a postdoc of the group at HITS (Martin Sparre).
The goal of this project is to study the effect of minor mergers on the
star formation rate of galaxies at different redshifts. This is important in
the context of galaxy formation because high star formation rates in galaxies 
at high redshifts remain unexplained. So far, there are several hypothesis 
supporting this picture, namely, accretion of cold star forming gas through 
infalling filamentary flows which are able to cross trough the shock-heated 
regions around galaxies and penetrate towards the centre. A second hypothesis 
consists of major merger events, where tidal forces compress the gas and 
trigger star formation. Finally, we propose to study a third scenario, that is 
not exclusive with the other two, but offers an appealing alternative as minor 
merger events are very common and not as potentially destructive as major 
mergers.

%.........................................................................
%FIGURE 3: Star formation rate
\begin{figure}[!htbp]
\centering

  \includegraphics[trim = 0mm 0mm 0mm 0mm, clip, keepaspectratio=true,
  width=0.57\textheight]{./figures/SFR.pdf}
  
  \caption{\small Star formation history for a galaxy in the Auriga project.
  The peaks are associated to star bursts induced by major or minor mergers.}

  \label{fig:SFR}

\end{figure}
%.........................................................................

For the purpose of this project, we use a set of zoom cosmological simulations 
(the Auriga project), where several galaxies have been simulated with very high
resolutions from sub-boxes of the Illustris simulation (see more details in
\citet{Grand16})). So far, several analysis scripts have been developed using 
the library MergerZoomAnalysis\footnote{
\url{https://bitbucket.org/martinsparre/mergerzoomanalysis}} of Martin Sparre. 
For example, in figure \ref{fig:SFR} it is shown the star formation history of 
one of the simulated galaxies, where the peaks correspond to star bursts induced 
by minor or major mergers. My current objective is then to develop a method to 
track minor merger events and correlate them with star bursts in the star 
formation history.


%==============================================================================
\bibliographystyle{latex/mn2e}
\renewcommand{\bibname}{8\ \ \ \ Bibliography}
\small
\bibliography{references2.bib}
%==============================================================================



\end{document}

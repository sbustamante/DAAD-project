\documentclass[a4,useAMS,usenatbib,usegraphicx,12pt]{article}
%External Packages and personalized macros
\include{latex/macros}
 
\title{{\textbf{Extension Report}}\\ 
				DAAD PhD Scholarship\\ 
				\color{black}\rule{15cm}{0.5mm}}
\author{Sebastian Bustamante Jaramillo}
\date{}
  
\begin{document}
\maketitle

\tableofcontents
 
\newpage 

%============================================================================== 
\section{Report}
%============================================================================== 

Since my arrival in Germany to the present day it has been almost one year. This
time can be split-up into two periods. The first four months, from 06.15 to 09.15,
correspond to the German course in Berlin. Finally, the second period, from 10.15 
to 04.16, correspond to my arrival in Heidelberg for starting my PhD.

\

Regarding the first period, there is not much academic progress to report on as 
I was focused on learning German. The only two exceptions perhaps were a 
first-author paper I pu\-blished on 08.15 \citep{Bustamante15}
\footnote{\url{http://mnras.oxfordjournals.org/content/453/1/497}} and a invited
talk I gave at AIP (Leibniz-Institut f\"ur Astrophysik Potsdam) about the 
results of the paper. Nevertheless, all the results reported in this work were 
already developed during my studies in Colombia and the activities during my 
stay in Berlin were limited to corrections of referee reports.

\

For the second period there are several things to report on. These can be split-up 
in 3 categories: general, corresponding to activities such as taking lectures 
and seminars, getting the admission to the university, etc. Major project, which 
is related to the main project I am developing with my supervisor, and finally, 
a minor project developed with a postdoc from the group.


%------------------------------------------------------------------------------
\subsection{General}
%------------------------------------------------------------------------------ 

After my arrival in Heidelberg, the first step was to enrol to the University as
a PhD student, however the assessment emitted by the HGFSP office (Heidelberg 
Graduate School of Fundamental Physics) was to take preparatory studies before 
being admitted as PhD student (this was because I do not hold a master's degree), 
which means I had to take two core lectures from the master. Nevertheless, this 
assessment was out of date as it corresponds to my old application in 2013. Since 
then, I had taken 1 semester of master studies in Colombia and published a 
first-author paper. Unfortunately, I had not brought documents to prove this, so
I had to wait for someone to bring them to me from Colombia. In the meantime,
it was decided by my Thesis Committee that I should take one of the core lectures 
during the first semester, so I decided to take Theoretical Astrophysics.

At the end of the semester, I finally got the documents, so I immediately 
submitted to the HGFSP office and one week later I was officially admitted as a
PhD student in the University of Heidelberg. It is worth mentioning that I also
finished the lecture, which included homework and a final examination.

\

Although the astronomy PhD program is officially hosted by the HGFSP, the 
associated activities are organized by the MPIA (Max Planck Institute for 
Astronomy) within the framework of the IMPRS program (International Max Planck
Research School), which is a structured program, and for that reason, there are 
several activities that the student members are required to participate in. 
Among these activities are the IMPRS literature seminar and the first-year 
student retreat. The seminar corresponds to a weekly session where every student 
must give a short talk about an assigned research paper. In my case, I attended 
this seminar during the first semester and also gave a talk about diffusive 
shock acceleration\footnote{\url{https://www.youtube.com/watch?v=L7OVGV6XJ0k}}.
About the first-year retreat, it consists of a social integration for new PhD
students and a visit to some astronomy institute in some city out of Germany.
The last retreat took place on 13.04.16 (four days) in Budapest, Hungary, where
we visited the Konkoly Observatory.


%------------------------------------------------------------------------------
\subsection{Major Project}
%------------------------------------------------------------------------------

The initial project I submitted in 2014 when applying for the scholarship was
about the study of the gaseous cosmic web in the context of the state-of-art 
hydrodynamical simulations using AREPO code (e.g. Illustris simulation\footnote{}).



%------------------------------------------------------------------------------
\subsection{Minor Project}
%------------------------------------------------------------------------------

Additionally, a minor project has been going on during the previous semester, 
which is being developed with a postdoc from the group at HITS (Martin Sparre).
The objective of this project is to study the effect of minor mergers on the
star formation rate of galaxies at different redshifts. For this purpose, a set
zoom cosmological simulations (the Auriga project \citep{Grand16}) are used.


\newpage
%==============================================================================
\bibliographystyle{latex/mn2e}
\renewcommand{\bibname}{8\ \ \ \ Bibliography}
\small
\bibliography{references2.bib}
%==============================================================================



\end{document}

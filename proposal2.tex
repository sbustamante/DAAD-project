\documentclass[a4,useAMS,usenatbib,usegraphicx,12pt]{article}
%External Packages and personalized macros
\include{latex/macros}
 
\title{{\textbf{Research Proposal}}\\ 
				Modeling supermassive Black Holes in hydrodynamical simulations of
				galaxy formation\\ 
				\color{black}\rule{15cm}{0.5mm}}
\author{Sebastian Bustamante Jaramillo}
\date{}
  
\begin{document}
\maketitle
\begin{center}
\includegraphics[trim = 0mm 2.0cm 0mm 2.0cm, clip, keepaspectratio=true,
width=0.7\textheight]{Presentation1.png}
\tiny{Artist representation binary black holes https://astronomynow.com/2015/09/22/pairs-of-supermassive-black-holes-in-galaxies-may-be-rarer-than-previously-thought/}
\end{center}
\tableofcontents
 
\newpage 

%============================================================================== 
\section{General Information}
\small
\subsection*{Information of the Applicant}
\begin{tabular}{L!{\VRule}R}
\bf Name		& Sebastian Bustamante Jaramillo\\
\bf Degree		& B.Sc. in Physics, Universidad de Antioquia\\
\bf Position	& PhD student. Heidelberg Institute for Theoretical Studies, Heidelberg University\\
\bf Birthday	& { 20$^{th}$ June, 1990}\\
\bf Nationality & Colombian\\
\bf Address	& Peterstaler Str. 96, 69118 Heidelberg\\
\bf E-mail 1	& macsebas33 \textit{at} gmail.com (personal)\\
\bf E-mail 2	& sebastian.bustamante \textit{at} h-its.org (academic)\\
\end{tabular}

\vspace{10pt}

More detailed information of the applicant can be found here \url{http://goo.gl/BPZGzK}

\vspace{15pt}  

\subsection*{Information of the Project}
\begin{tabular}{L!{\VRule}R}
\bf Title		& \bf Modeling supermassive Black Holes in hydrodynamical simulations of
				galaxy formation\\
\bf Field		& Cosmology, Astrophysics, Physical Sciences \\
\bf Advisor	& Professor Volker Springel. Heidelberg Institute for Theoretical Studies (HITS) 
\& University of Heidelberg, Germany \\
\bf University	& University of Heidelberg, IMPRS PhD program \\
\bf Time Frame	& 3 years \\
\end{tabular}
\normalsize
%==============================================================================


%==============================================================================
\section{Abstract}
%==============================================================================


The modelling of supermassive black holes in the center of galaxies is an important 
enterprise to be carried out as they play a central role in the current understanding
of start formation theories. However, simulating such processes in a cosmological 
context is not an easy task due to the limited numerical resolution achieved by 
current state-of-art computing systems. One of our objective is to show that at 
typically employed numerical resolutions, the orbits of sink particles used to 
represent supermassive black holes in simulations can be quite unreliable, as a 
result of two-body effects, fluctuating gravitational potentials and noisy 
dynamical friction forces. We want to test several proposals from the literature 
to improve this treatment, and suggest a new one as well. We shall use the 
improved methods to investigate how black hole recoil kicks affect the growth of 
the black hole population when black holes return to the centers of halo potentials 
on realistic timescales.


\newpage

%==============================================================================
\section{Motivation}
%==============================================================================

\subsection*{First phase}

Supermassive black holes are ubiquitously observed in the centers of galaxies, 
and they play a critical role in current theories for galaxy formation, where 
they are supposed to suppress star formation in large galaxies by injecting 
energy into the gas. We hence want to simulate the growth of these black holes 
and the associated co-evolution with the galaxy when studying numerical models 
of galaxy formation. In reality, it is believed that the black holes experience 
dynamical friction against the background of dark matter and stars, and possibly 
also through gas-dynamical processes, making them spiral in to the centers of 
galaxies on a reasonably short timescale. Only when they are positioned there, 
they can efficiently influence the whole galaxy and grow rapidly. 

\

After galaxy mergers, the remnant (merged) black hole will return to the center 
through these friction processes, after being kicked out through asymmetric 
emission of gravitational waves. To properly estimate how long the growth/feedback 
may be weak/interrupted after a merger (because the center is not yet found again), 
and how many free floating massive black holes there may be, the orbit of the 
black holes needs to be followed reasonably accurately.

\subsection*{Second phase}

Whenever supermassive black holes merge during galaxy mergers, they emit a burst 
of gravitational wave radiation. This happens in an symmetric way such that there
is a quite large recoil, kicking the BH out of the centre. It could even happen 
that the BH leaves the remnant halo entirely, but usually, it probably returns 
after some time. During this period, the galaxy may then grow unimpeded by the BH. 
We would like to find out how strongly predictions by galaxy formation are 
modified when the BH recoil kicks are accounted for. To this end, one can apply 
fitting functions produced by numerical relativity simulations to kick BH merger 
remnants depending on their mass ratio, whenever this happens in a cosmological 
simulation (see for example Sijacki et al., MNRAS, 2011, 414, 3656, arxiv:1008.3313). 
Combined with a treatment of BH friction, the BHs are then expected to return to 
the centers after a finite time, so that these effects can be studied in modern 
simulations of galaxy formation.


%==============================================================================
\section{Problem}
%==============================================================================


Following the orbit of supermassive black holes in cosmological simulations is 
not readily possible, as the masses of dark matter and star particles in N-body 
simulation are very much larger than in reality and similar to the mass of the 
black hole. As a result, two-body scattering effects will try to 'heat-up' the 
central black hole particle and prevent it from experiencing proper dynamical 
friction, or in other words, the black hole will not return to the center of the 
potential by itself under these conditions. Current simulation models therefore 
usually employ non-physical tricks to 'glue' the black hole particle to the 
potential minimum, for example by searching for the smallest black hole potential 
value among neighbors around the black hole, and then simply positioning the black 
hole particle to this minimum. This is for example done by the Illustris and 
Gigagalaxy projects. While this prevents that the central black hole particle is 
lost, it also prevents that the above questions can be studied, and potentially 
one also introduces severe inaccuracies in the efficiency with which black holes 
can grow.


%==============================================================================
\section{Objectives}
%==============================================================================

The objectives of the project are:

\begin{itemize}

\item[\checkmark] Improving and proposing semi-analytical methods to follow 
accurately the orbit of supermassive black holes in cosmological simulations,
where dynamical friction is accounted for.

\item[\checkmark] Studying the effect of dynamical friction on the properties
of merging galaxies, e.g. how the growth and feedback of the black hole is 
strengthened, weakened or even interrupted during merging processes.

\item[\checkmark] Making statistical predictions, in a cosmological context, 
of gravitational wave emissions produced by coalescing black hole pairs.
This is very much important for current and future experiments such as LISA.
\footnote{Laser Interferometer Space Antenna for observing gravitational waves
\url{https://www.elisascience.org/}.}


\end{itemize}


%==============================================================================
\section{Methodology}
%==============================================================================


One approach for improving the modelling would be to augment the equation of 
motion for the black hole particle by an explicit dynamical friction force. 
This can be done in different ways.

\begin{itemize} 
\item One can try to add a suitably modified version of Chandrasekhar's dynamical 
friction formula, similar to how this is attempted in Tremmel et al. (2015, MNRAS, 
451, 1868, arxiv:1501.07609). This involves some assumptions and technical 
approximations. In the formulation of Tremmel, they arranged it such that the 
force becomes ever weaker in the limit of infinite resolution, so that one argue 
it is a correction for the finite resolution of real-work simulations. However, 
the tests presented in the paper are not fully convincing and it is still unclear 
whether this method works sufficiently well in practice (e.g. for galaxy merger 
simulations, and for cosmological simulations of galaxy formation).

\item Because it is not really clear whether Chandrasekhar's formula applies 
well when the BH is already close to the centre of the galaxy (where stars need 
to be ejected from the 'loss-cone' and interactions with gas play a very important 
role), one may instead also conjecture different models. For example, if we assume 
that the BH is brought back efficiently to the potential minimum if it is displaced 
from the centre, we can construct an 'optimum' friction force as follows: At the 
centre, the gradient of the potential is zero by definition, so that it can be 
approximated as quadratic to first order. The BH is hence expected to carry out 
harmonic oscillations around the centre. We can now try to define an optimum damping 
force that eliminates the oscillation on the shortest possible timescale (if the 
friction is too large, the motion will be 'overdamped', and one takes longer to the 
centre than for a suitably smaller force). We can estimate the potential around the 
centre as $g(x) \sim \phi_0 + (1/2) (d^2\phi/dx^2) x^2$, and the second derivative
along one direction can be estimated from Poisson's equation as $(d^2\phi/dx^2) 
\sim (1/3) (4 \pi G \rho)$, where $\rho$ is the local total mass density. The 
harmonic oscillator's equation of motion will be $x'' = - \omega^2 x = - (dg/dx) x$, 
meaning that the oscillation frequency is $\omega = \sqrt{(4 \pi G \rho)/3}$. 
For a critical damping we would then need to add a friction force as $x'' = 
-\omega^2 x - k x'$, where $k = 2 \omega$. In this case, we expect the orbit to 
decay on the local free fall timescale.


\end{itemize}


%==============================================================================
\section{Work Steps}
%==============================================================================

\subsection*{First phase}

The second idea presented in the methodology section is already implemented in 
the AREPO code in a first version, through the switch BH\_FRICTION. This 
implementation tries to determine a reference velocity of the potential minimum 
by measuring its position in the environment of the BH, and then taking finite 
differences to estimate this velocity (this is needed because for applying the 
friction force antiparallel to the velocity one needs to know the local reference 
frame). The oscillation frequency is estimated from the total matter density, 
and from this a friction force is estimated antiparallel to the velocity against 
the local frame of rest, with a tunable coefficient in front of order unity.

\begin{itemize} 
 
\item  As a first step, the implemented equations need to verified. This also requires that I
explain you the many (in part quite complicated) technical aspects of the code that play a
role here
\item  We should then test the effects this implementation has, and how well it is working, in a
number of situations:

\begin{itemize}


\item Simulations of Milky Way sized galaxies. Here, an important diagnostic would be to
develop a plot script that shows the distance of BHs relative to the potential minima of
the
corresponding halos. This can be done based on the group catalogues, I think. First, one
could look at the current state of affairs in some of the finished production simulations of
Rob and Ruediger. Next, one should repeat one or two of these runs by disabling the BH
centering that had been used (to see how bad the displacements get if one doesn’t try to
correct them). Finally, one should try one or several variants of the friction treatments
above, starting with method (2) and later also with method (1).

\item Toy simulations of an isolated halo where one injects a BH with some velocity and
impact parameter to study the decay of the orbit. Again, one should compare different
friction treatments (ideally also at different resolution) with the results where no such thing
is applied.

\item Finally, one repeats the test of (a) for a cosmological simulation, where one can
compile statistical results, e.g. about the distribution function of the average distances of
BHs to their halo centers.

\end{itemize}

\end{itemize}

\subsection*{Second phase}

- One first needs to compile recent results from numerical relativity simulations to get the
most up-to-date fitting function for the strength of the recoil kicks as a function of BH
mass ratio.
- One then needs to implement an option in AREPO that adds a kick with the
corresponding strength after the merger of two black holes.
- Then, one repeats a cosmological simulation of galaxy formation with the kicks enabled,
and compares to the equivalent one without kicks.
- The quantities to focus on are, in particular, the galaxy luminosity function at the bright
end, the BH mass function, and the color distribution of galaxies. Also interesting would
be histograms that quantify the distance of BHs from the centre, the time of return to the
centre, the number of free floating BHs, etc.
- Implications for the merger rate of BHs, and the distribution of the mass ratios, would
also be particularly interesting as this is directly related to the statistics of gravitational
wave emission events.



%==============================================================================
\section{Schedule}
%==============================================================================

\begin{table}[h]
\begin{flushleft}
\begin{center}
  \begin{tabular}{l  l} \hline\hline
	\centering\textbf{Year} & \textbf{Goals} \\ \hline
	%First year
	First  
	& \tabitem Identifying a set of existing \texttt{AREPO} simulations suitable 
	for our studies. \\
	& \tabitem Applying web finding schemes (T-web and V-web) to the simulations 
	for\\
	& \ \ \ \ quantifying structures in the gaseous cosmic web, i.e. voids, walls, 
	filaments\\
	& \ \ \ \ and clusters.\\
	& \tabitem Evaluating properties of found structures at different redshifts.\\
	\\
	%Second year	
	Second
	& \tabitem Studying by means of high resolution simulations the impact of the 
	gaseous\\
	& \ \ \ \ cosmic web on specific galaxy evolution processes.\\
	\\	
	%Third year	
	Third
	& \tabitem Comparing with available observational data of the cosmic web.\\
	& \tabitem Deriving new observable from our theoretical studies.\\ 
	
	\hline\hline
  \end{tabular}  
\end{center}
\end{flushleft}
\end{table}

\newpage
%==============================================================================
\bibliographystyle{latex/mn2e}
\renewcommand{\bibname}{8\ \ \ \ Bibliography}
\small
\bibliography{references.bib}
%==============================================================================



\end{document}
